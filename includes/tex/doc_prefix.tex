\title{Schwa to /a/: Development of Unstressed Vowels in L2 Spanish}
\author{Robert Esposito}
\organization{Department of Spanish and Portuguese, Rutgers University}
\email{rme70@scarletmail.rutgers.edu}


\maketitle

\begin{abstract}
Native (L1) English second language (L2) Spanish learners acquiring the Spanish vowel system are tasked with adjusting entrenched vowel categories and morphophonological rules. For example, allophonic variation of stressed and unstressed vowels in English is not present in Spanish, but the triggering environments for the variation is present in Spanish. As such, learners of Spanish must come to ignore those triggering environments during perception and production. The present study examines the production of unstressed Spanish /a/ by 10 L1 English L2 Spanish learners during a seven-week domestic immersion program in a semi-longitudinal study. Preliminary acoustic analysis revealed that within the first week, participants fronted and lowered their production of unstressed /a/, assumed here to correspond to less centralization and more target-like production. The findings are situated within current models of second language speech learning and provide often-sought longitudinal data on an understudied phenomenon.
\end{abstract}

\keywords{phonetics, phonology, second language acquisition, Spanish, English}

